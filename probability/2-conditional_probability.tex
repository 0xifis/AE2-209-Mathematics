\section{Conditional probability} % (fold)
\label{sec:conditional_probability}

The conditional probability of $A$ given $B$ is written as

\begin{equation}
  P(A|B) = \frac{P(A\cap B)}{P(B)}
\end{equation}

Some properties of conditional probabilities.

\begin{subequations}
  \begin{equation}
    P(A|B) \geq 0
  \end{equation}
  \begin{equation}
    P(S|B) = 1
  \end{equation}
  \begin{equation}
    \text{If A and B are disjoint} \rightarrow P(A \cup B|C) = P(A|C) + P(B|C)
  \end{equation}
  \begin{equation}
    P(\overline{A}|B) = 1 - P(A|B)
  \end{equation}
\end{subequations}


%--------------------------------------------------
\subsection{Independent Events} % (fold)
\label{sub:independent_events}

$A$ and $B$ are independent if
\begin{equation}
  P(A\cap B) = P(A)P(B)
\end{equation}
$A$, $B$ and $C$ are co-independent if
\begin{equation}
  P(A\cap B\cap C) = P(A)P(B)P(C)
\end{equation}
% subsection independent_events (end)
%--------------------------------------------------

%--------------------------------------------------
\subsection{Law of Total Probability} % (fold)
\label{sub:law_of_total_probability}

Let $B$ be an event and $A_i$ be disjoint events.

\begin{equation}
  S = A_1 \cup A_2 \cup A_3... \cup A_n
\end{equation}

\begin{equation}
  P(B) = \sum^n_{i=1}P(B|A_i)P(A_i)
\end{equation}
% subsection law_of_total_probability (end)
%--------------------------------------------------

%--------------------------------------------------
\subsection{Bayes' Theorem} % (fold)
\label{sub:bayes_theorem}

Let $A_i$ be disjoint events and B be another event. Then,

\begin{equation}
  P(A_j|B) = \frac{P(B|A_j)P(A_j)}{\sum_{i=1}^nP(B|A_i)P(A_i)}
\end{equation}

% subsection bayes_theorem (end)
%--------------------------------------------------
% section conditional_probability (end)
