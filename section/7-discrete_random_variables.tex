\section{Discrete Random Variables} % (fold)
\label{sec:discrete_random_variables}

%--------------------------------------------------
Let $X$ be a discrete random variable that takes values in the discrete set $S = {x_1, x_2, ...}$. Let $P(X = x_i) = p_i$, that is, the probability that $X$ takes the value $x_i$ is equal to $p_i$. Then any set of values $p_i$ satisfying

\begin{subequations}
  \begin{equation}
    p_i \geq 0
  \end{equation}
  \begin{equation}
    \sum_{i}p_i = 1
  \end{equation}
\end{subequations}
%--------------------------------------------------

%--------------------------------------------------
\subsection{Binomial Distribution} % (fold)
\label{sub:binomial_distribution}

A binomial distribution is used when we have a trial that
\begin{enumerate}
  \item either  succeeds with probability $p$ or fails with probability $1-p$
  \item is repeated $n$ times, \textbf{independently}
\end{enumerate}

For n trials, for the probability of $X$ successes has the following probability distribution

\begin{equation}
  P(X=m) = \binom{n}{m}p^m(1-p)^{n-m}
\end{equation}
% subsection binomial_distribution (end)
%--------------------------------------------------

%--------------------------------------------------
\subsection{Poisson Distribution} % (fold)
\label{sub:poisson_distribution}

A Poisson distribution is used to model events that occur randomly and independently in space and time.

Assuming events occur randomly at a rate $\lambda \geq 0$,
\begin{equation}
  P(X=n)=\frac{e^{- \lambda t}(\lambda t)^n}{n!}
\end{equation}
% subsection poisson_distribution (end)
%--------------------------------------------------
% section discrete_random_variables (end)
